\section{Conclusion}

Dans ce chapitre, nous avons présenté les détails de la conception de la solution que nous avons proposé.
Il s'agit d'un système à deux composantes principales : 
un modèle d'\gls{asr} et un modèle de traduction.
Pour chacune de ces composantes, nous avons expliqué la procédure suivie 
pour l'acquisition et la préparation des données.

Dans le cas du modèle d'\gls{asr}, nous avons trouvé que les données ne sont pas suffisantes 
pour entraîner un modèle de qualité.
Nous nous sommes donc contentés de la création d'un corpus de données qui peut être exploité par des futurs travaux.

Pour le cas de la traduction, nous avons créé un corpus synthétique de taille suffisante pour entraîner un modèle.
Nous avons présenté les différentes décisions que nous avons pris pour la création et l'entraînement de ce modèle.
Ces décisions incluent le choix de l'architecture et de l'algorithme d'entraînement, 
les valeurs des hyperparamètres, les métriques à mesurer et la possibilité d'optimisation des hyperparamètres.

Dans le chapitre suivant, nous présentons notre réalisation de la solution conformément à la conception proposée.


