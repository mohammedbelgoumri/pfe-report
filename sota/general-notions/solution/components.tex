\subsection{Structure proposée pour la solution}

Au lieu de mapper l'audio erroné directement à l'audio corriger,
nous proposons un système qui enchaîne trois transformations.
La première transforme \(S_A\) en une représentation textuelle \(T_A\)
La deuxième produit un texte corrigé \(T_C\) à partir de \(T_A\).
La dernière étape est de produire \(S_C\) à partir de \(T_C\) (voir la Figure~\ref{fig.detail-system}).

\begin{figure}[hbt]
    \centering
    \includegraphics[width=\linewidth]{assets/images/detail.png}
    \caption{Étapes intermédiaires du système proposé.}
    \label{fig.detail-system}
\end{figure}

L'étape finale de cette procédure est la plus simple et peut être considérée comme résolue.
La tâche en question est tout simplement de produire un audio à partir d'un texte.
Il s'agit de synthèse vocale.
Plusieurs solutions existent pour cette tâche, y compris des solutions open-source~\cite{Tan_et_al._2022}.
Nous nous concentrerons donc sur les deux premières étapes.

La première partie de notre système réalise une transcription automatique de l'audio.
C'est une tâche de \gls{asr}, 
avec la complexité supplémentaire que la parole en question ne suit pas les lois usuelles du langage.

La seconde partie est sans doute la plus complexe.
C'est la partie centrale de notre système qui réalise directement la correction.
Pour ce projet, nous avons choisi de traiter le problème comme un problème de \gls{mt}.
Autrement dit, le texte erroné et le texte corrigé 
sont considérés comme s'ils appartiennent à deux langues différentes,
entre lesquelles nous voulons réaliser une correspondance.

Ces deux étapes sont deux exemples de tâches de \gls{nlp}.
Plus précisément, elles sont des tâches de modélisation \gls{s2s}.

Dans le reste de ce document, nous détaillerons l'état de l'art de ces deux tâches.
Le chapitre suivant présente en détail le problème général de modélisation \gls{s2s},
ainsi que les différentes approches de \gls{dl} 
qui ont été proposées dans la littérature pour le résoudre.
Cela a pour but de choisir une approche de \gls{dl} 
pour implémenter les deux premières composantes de notre système.

Celui qui suit aborde plus spécifiquement les problèmes d'\gls{asr} et de \gls{mt}
dans le contexte où elles peuvent être appliquées dans notre système.
Nous y présentons une revue de la littérature pour le cas particulier des tâches d'intérêt.
Dans le cas où elles existent, nous présentons des cas d'applications similaires à notre problème.
