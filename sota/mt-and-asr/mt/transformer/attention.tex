\subsection{Traduction automatique à base de transformeur}
\label{subsec.nmt-transformer}

Le transformeur tel que nous l'avons présenté dans section~\ref{sec.transformers}
peut être utilisé pour la \gls{mt}.
L'utilisation de cette architecture pour la correction automatique de la parole aphasique
a été explorée par \cite{Misra_Mishra_Gandhi_2022} qui ont employé un transformeur pré-entraîné
appelé \gls{t5}.
En effet, \cite{attention} l'ont appliqué à cette même tâche.
Ayant déjà introduit l'architecture et le principe de fonctionnement du transformeur,
nous nous concentrerons sur les particularités de son application à la \gls{mt}.

\subimport{}{embedding}
\subimport{}{evaluation}
\subimport{}{decoding}