\section{Traduction automatique à base de règles}
\label{sec:rbmt}

La \acrfull{tabr} est historiquement le premier paradigme de \acrshort{ta}. 
Étant apparue pendant les années 1950s, 
elle resterait l'approche dominante de \acrshort{ta} jusq'aux 1980s~\cite{routledge}.

Comme son nom l'indique, la \acrshort{tabr} est basée sur des règles de traduction explicites,
qui sont généralement créées manuellement à partir de connaissances linguistiques sur la \acrfull{ls} et la \acrfull{lc}. 
Les règles en question peuvent être d'ordre lexical (i.e des dictionnaires),
syntaxique (i.e des grammaires) ou sémantique.

% Ces connaissances prennent souvent la forme de règles de traduction 
% qui peuvent être lexicales (i.e. des dictionnaires), 
% syntaxiques (i.e. la grammaire)
% ou sémantiques.

% La \acrshort{tabr} passe par deux phases: l'analyse du texte 


