\subsection{Généralités sur le cerveau}

Pour mieux comprendre l'aphasie en général et celle de Broca en particulier, 
il convient de commencer avec le cerveau.
Il s'agit du système biologique le plus complexe connu~\cite{}.
Avec le cervelet et le tronc cérébral, il forme l'encéphale (voir Figure~\ref{fig. brain}).
Le cerveau se charge du traitement des flux nerveux sensoriels et moteurs.
Il est aussi le siège des hautes fonctions cognitives comme l'inférence logique, l'émotion 
et --- crucialement pour notre étude --- le traitement du langage~\cite{}.

\begin{figure}[htb]
    \begin{center}
        \includegraphics[width=.8\textwidth]{assets/images/brain.png}
    \end{center}
    \caption{Encéphale humain}
    \label{fig. brain}
\end{figure}

Le cerveau est composé de deux hémisphères ; chacun desquels se divise en lobes : 
frontal, temporal, pariétal et occipital (voir Figure~\ref{fig. lobes}).
La surface du cerveau s'appelle le ``cortex cérébral''.
Il présente plusieurs circonvolutions qui augmentent considérablement sa surface.
Le cortex cérébral est divisé en régions fonctionnelles que nous appelons ``aires''
(voir Figure~\ref{fig. brain-areas}).
Le travail de Dr.~Broca sur le cas de M.~Leborgne est largement reconnu comme l'origine de cette division.


\begin{figure}[htb]
    \begin{center}

        \begin{subfigure}{.38\linewidth}
            \includegraphics[width=\linewidth]{assets/images/lobes.png}
            \caption{Lobes du cerveau~\cite{}}
            \label{fig. lobes}
        \end{subfigure}
        \begin{subfigure}{.52\linewidth}
            \includegraphics[width=\linewidth]{assets/images/areas.png}
            \caption{Aires du cortex cérébral~\cite{}}
            \label{fig. brain-areas}
        \end{subfigure}
        
        
    \end{center}
    \caption{Division morphologique et fonctionnelle du cerveau.}
\end{figure}

Une autre division importante et due à l'anatomiste Allemand Korbinian Brodmann.
Elle se base sur l'organisation cellulaire des neurones pour segmenter le cortex cérébral en 52 régions 
aussi nommées ``aires''.
En dépit d'avoir une définition morphologique, 
les aires de Brodmann sont largement alignés sur les aires fonctionnelles 
de la Figure~\ref{fig. brain-areas}~\cite{Brodmann_2007}.

Dans cette étude, nous donnons un intérêt particulier aux aires de Brodmann 44, 45 et 39, 40.
En effet, les deux premiers correspondent à la région du cerveau de M.~Leborgne où Dr.~Broca trouva la lésion.
Elles portent donc son nom : aire de Broca.