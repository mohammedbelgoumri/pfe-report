\section{Architecture encodeur--décodeur}

Les lacunes des \glspl{mlp} sont en large partie due au traitement séparé des parties des séquences.
Dans la production de l'élément (ou bloc) courant de la sortie,
un \gls{mlp} se base uniquement sur l'élément (ou bloc) correspondant de l'entrée.
L'équation~\ref{eq.mlp-seq-prod} l'illustre pour une entrée \(x = (x_1, x_2, \cdots, x_n)\)
et une sortie \(y = (y_1, y_2, \cdots, y_m)\).
\begin{equation}
    \label{eq.mlp-seq-prod}
    y_j = f(x_i, x_{i+1}, \cdots, x_{i+\ell}) \qquad 1 \le j \le m
\end{equation}
En plus de l'hypothèse implicite de l'existence d'une telle correspondance,
cela suppose que les éléments d'une séquence sont complétement indépendant l'un de l'autre.
Cette dernière hypothèse n'est presque jamais vérifiée. 

Une façon naturelle de combler ces lacunes est d'abandonner le traitement par bloc de l'entrée.
Tout élément de la séquence de sortie est considéré comme fonction de la séquence d'entrée toute entière.
L'équation~\ref{eq.enc-dec} montre cette approche sur le même exemple.
\begin{equation}
    \label{eq.enc-dec}
    y_j = f(x) = f(x_1, x_2, \cdots, x_n) \qquad 1 \le j \le m
\end{equation}

% tenir compte de toute l'entrée à la pro
% produire chaque élément de la sortie en partant de l'entrée toute entière. 