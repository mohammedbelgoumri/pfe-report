\section{Étiologie et épidémiologie de l'aphasie}

Les \Acrshort{avc} sont la première cause d'aphasie~\cite{Hallowell_2017}. 
Ils représentent \(75\%\) des cas.
Le traumatisme crânien en provoque \(5\%\) 
et les \(20\%\) restants se partagent entre les autres causes.
L'age est un facteur de risque très important pour les \Acrshort{avc},
il l'est donc également pour l'aphasie. 
En effet, l'age moyen des individus Français atteints de l'aphasie et 73 ans.
\(75\%\) parmi eux sont âgés de plus de 65 ans dont \(25\%\) dépassent les 80 ans~\cite{press}.

Il est difficile d'estimer l'incidence et la prévalence globales de l'aphasie.
Ceci et due au manque de données dans la majorité des pays du monde.
Selon l'association nationale de l'aphasie~\cite{Home}, 2 millions Américains en souffrent, soit \(0.6\%\).
En France, ce chiffre est de l'ordre de 300000 cas totaux et 30000 cas chaque année~\cite{press}.
Ceci donne une prévalence de \(0.44\%\) et un taux d'incidence \(0.044\%\).

\(33\%\) des \Acrshort{avc} résultent en une aphasie~\cite{press}.
Un an après l'\Acrshort{avc}, \(13\%\) des patients développent une aphasie de Broca.
