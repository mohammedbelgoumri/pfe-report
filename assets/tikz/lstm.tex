\begin{circuitikz}
    \ctikzset{bipoles/cuteswitch/thickness=.5,bipoles/cuteswitch/shape=circ}
    % cell outline
    \draw[fill=green!10, rounded corners=15pt](1, -3) rectangle (13, 3);

    % nodes
    \node (x) at (-1,0) {\(x\)};
    \node[inner sep=0pt] (add1) at (2,0) {\(\oplus\)};
    \node[rectangle, draw] (candidate) at (2, 2) {\(\widetilde{C}\)};
    \node[inner sep=0pt] (add2) at (6,0) {\(\oplus\)};
    \node[circle, draw] (state) at (8, 0) {\(C\)};
    \node[circle, draw, fill, inner sep=0mm] (branch) at (6, -2){};
    \node[circle, draw] (activation) at (10, 0) {\(\phi\)};
    \node (h) at (14, 0) {\(h\)};

    % connections
    \draw[-{Latex}] (x) to (add1);  
    \draw[-{Latex}] (add1) to (candidate);
    \draw[-{Latex}] (candidate) to[cosw, l_=\(i\)] (6, 2) to (add2);
    \draw[-{Latex}] (add2) to (state);
    \draw[-{Latex}] (state) to (8, -2) to (branch) to (2, -2) to (add1); 
    \draw[-{Latex}] (branch) to[cosw, l_=\(f\)] (add2);
    \draw[-{Latex}] (state) to (activation);
    \draw[-{Latex}] (activation) to[cosw, l_=\(o\)] (h);
\end{circuitikz}    
